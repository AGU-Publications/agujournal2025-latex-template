\documentclass[published]{agujournal2025_4.22.25}


\begin{document}


\journalname{Journal name}

% Your title can be multiple lines long.
\title{Title of your paper}


\authors{%
                  Your First Author Name\affil{their author footnote number, e.g., 1 or 1,2}\thanks{their additional affilation address and/or funding information},
	          % Repeat the above for each author, with commas between authors, e.g.:
                  Your Second Author Name\affil{their author footnote number, e.g., 2 or 2,3}\thanks{their affiliation address and/or funding information}
	      }
	      

% Repeat for all authors:	      
\affiliation{1}{their affiliation when they (co-)wrote the paper} % First argument is their footnote number, as above.



% Corresponding author. Do not prepend with ``SI Corresponding author: '' in published version.
\authoraddr{%
                      % e.g., their full name, their department, their academic institution, their building, their City, State, Zip, Country. (theiremail\@ their institution)
                      Corresponding author name, address, etc.
                   }



\authorrunninghead{Ignored.} % Ignored.

% Key points.  
%   Takes three arguments.  Remember to specifiy them all, using {} for no item.
%
\keypoints%
% {}{}{}
%    % First key point, etc.
    {Key point 1}{Key point 2}{Key point 3}
%    {Summarize the main points and conclusions of the article}
%    {Each must be 140 characters or fewer with no special characters or punctuation and must be complete sentences}

% Call after \authors.

% Title running heads will be ignored, too:
\titlerunninghead{Ignored.}

% Key points was here

\maketitle


% Two options for abstracts: one with an abstract only and one with a plain language summary included.
%
%\begin{abstract}
%    Abstract text body.
%\end{abstract}


\bigskip  % Remove; for demo purposes only.
% OR:

\begin{abstract}
    Abstract text body with some more content.
    %
     \begin{plainlanguagesummary}
            Plain language summary text body with some more content.
     \end{plainlanguagesummary}
     %
     % Do not put more abstract text body here!
\end{abstract}


\section{Section Name}  % Or first section name if you have not introduction.

Section text.


\subsection{Subsection Name}

Subsection text.


\subsubsection{Subsubsection Name}


\paragraph{Paragraph Name}
Paragraph text.
%
% No paragraph space here unless really starting a new paragrah.

New paragraph.
%
% Use description environment for definition lists.
\begin{description}
    \item[First] First description.
    \item[Second] Second description
\end{description}

Section text.
% No paragraph space here unless really starting a new paragrah.
 \begin{equation}
 x^2=y^2 + z^2
 \end{equation}
Section text.
%
\begin{enumerate}
    \item List item text.
        \begin{enumerate} % Can nest lists.
          \item List item text.
          \item List item text.
        \end{enumerate}
    \item List item text.
\end{enumerate}
% No paragraph space here unless really starting a new paragrah.
Section text.
% No paragraph space here unless really starting a new paragrah.
\begin{itemize}
    \item List item text.
        \begin{itemize}
          \item List item text. 
          \item List item text.
        \end{itemize}
    \item List item text.
\end{itemize}
% No paragraph space here unless really starting a new paragrah.
Section text.

ShortCite CitekeyArticle \cite{CitekeyArticle}, ShortCiteA CitekeyBook \citeA{CitekeyBook}. 

\textsf{This is the sans serif font.}

 \texttt{This is the mono font.}
 
 % Footnotes will generate an error on the page and in the console.

% Program code:
\begin{verbatim}  
   This is verbatim text where macro names are shown rather than expanded, e.g., \LaTeX.
\end{verbatim}

Section text.



%% Figures can float away from where you insert them.
%%
%\begin{figure}[h]
%    \includegraphics[width=\textwidth]{figure-textwidth.png}
%    \caption{Figure caption.}
%\end{figure}
%
%% Figures too wide for the text width will go off the right edge.
%\begin{figure*}
%        \includegraphics[width=41pc]{figure-wide.pdf} 
%        \caption{Figure caption.}
%\end{figure*}%{fullwidthfloat}

Section text.


% You do not have to change the table numbers.  That's automagic.
%    RE: \settablenum{S1} % Change number for each table

% Tables that are wider than the text width will run off the right side.
% You can also make fancier tables.
\begin{table}
\caption{Table caption.}
\centering
\begin{tabular}{l c}
    \hline
        Run  & Time (min)  \\
   \hline
      $l1$  & 260   \\
      $l2$  & 300   \\
      $l3$  & 340   \\
    \hline
\end{tabular}
\tablenotetext{a}{Table footnote text here.}  %xxx ditto    %xxxx table footnote, not just footnote
\end{table}

Section text.


% Your Data Availability Statement is an unnumbered section right before the bibliography.
\section*{Open Research Statement}

Statement text.



% Bibliography
%\cite{*} % If you have uncited bibliography items.
%
\bibliography{wiley} % Replace ``wiley'' with your bibliography's file name.  Do not specify its extension, e.g., .bib.



% Appendix sections are numbered differently than article ones.
\appendix  % Put this first.

% Can repeat for each appendix.
\section{Appendix 1}
Appendix text.

% You can use subsection, etc.
\subsection{Appendix 1 Subsection}
Appendix text.



\end{document}





